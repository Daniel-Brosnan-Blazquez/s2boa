\section{Ingestion module for the SRA file}

This sections describes the ingestion module for inserting the SRA information received from the \acrshort{edrs}.

The associated ingestion processor is:

\begin{itemize} 

\item \textbf{s2boa.ingestions.ingestion\_slot\_request\_edrs.ingestion\_slot\_request\_edrs}
  
\end{itemize}

This module uses the following \acrshort{dim} signatures:

\begin{itemize} 

\item \textbf{SLOT\_REQUEST\_EDRS}: data corresponding to the slot request information associated to the EDRS service.

\item \textbf{COMPLETENESS\_NPPF\_XXX}: data corresponding to the definition of planning completeness used for analysis.
  
\end{itemize}

Where XXX is the corresponding satellite ID.

The figure \ref{fg:structure_ingestion_slot_request_edrs} shows a simplified diagram of the structure of events inserted (associated structure of values not included for simplicity).

\begin{figure}[H]
  \begin{center}
	\centering\includegraphics[width=150mm]{structure_ingestion_slot_request_edrs.png}
	\caption{Structure of events inserted by the ingestion module for the SRA file}
	\label{fg:structure_ingestion_slot_request_edrs}
  \end{center}
\end{figure}

The table \ref{tb:description_events_ingestion_slot_request_edrs} shows the description of the events inserted by the ingestion.

\begin{landscape}
\begin{longtable}{|M{0.15\linewidth}|M{0.05\linewidth}|M{0.10\linewidth}|M{0.10\linewidth}|M{0.15\linewidth}|M{0.15\linewidth}|M{0.15\linewidth}|}
\hline \textbf{Gauge name} & \textbf{Gauge system} & \textbf{DIM signature} & \textbf{Insertion mode} & \textbf{Description} & \textbf{Start} & \textbf{Stop} \\ \hline
\textbf{SLOT\_REQUEST\_EDRS} & XXX & SLOT\_REQUEST\_EDRS & ERASE\_and\_REPLACE\_per\_EVENT (insert) & Event for representing the \textbf{slot request for \acrshort{edrs}} & UTC value inside the Start\_Time node & UTC value inside the Stop\_Time node \\ \hline
\textbf{STATION\_SCHEDULE\_COMPLETENESS} & XXX & \- COMPLETENESS\_NPPF\_XXX & ERASE\_and\_REPLACE (insert\_and\_erase) & Event for representing the \textbf{expectation of the Station schedule} & UTC value inside the Start\_Time node & UTC value inside the Stop\_Time node \\ \hline
\textbf{DFEP\_SCHEDULE\_COMPLETENESS} & XXX & \- COMPLETENESS\_NPPF\_XXX & ERASE\_and\_REPLACE (insert\_and\_erase) & Event for representing the \textbf{expectation of the DFEP schedule} & UTC value inside the Start\_Time node & UTC value inside the Stop\_Time node \\ \hline
\caption{Table describing the events associated to the ingestion}
\label{tb:description_events_ingestion_slot_request_edrs}
\end{longtable}
\end{landscape}

\subsection{Ingestion details}

This section describes some ingestion details for inserting the data. In particular:

\begin{itemize} 

\item The correction of the generation time to avoid overriding data used for completeness analysis
  
\end{itemize}

\subsubsection{Correction of the generation time}

The generation time of the data extracted is one day before the validity start. This could be a problem as the processor of the ORBPRE files could override this data.

To solve this issue the processor changes the generation time to be the validity start.
